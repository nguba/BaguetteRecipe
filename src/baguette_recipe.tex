\documentclass[10pt,a4paper]{article}
\usepackage[utf8]{inputenc}
\usepackage[margin=1in]{geometry}
\usepackage{booktabs}
\usepackage{array}
\usepackage{siunitx}
\usepackage{longtable}
\usepackage{fancyhdr}
\usepackage{titlesec}
\usepackage{enumitem}
\usepackage{hyperref}

% Header and footer
\rhead{\textcolor{headerblue}{\textbf{Professional Baguette Formula}}}
\lfoot{\textcolor{headerblue}{Cold Bulk Fermentation Method}}
\rfoot{\textcolor{headerblue}{\textbf{\thepage}}}
\renewcommand{\headrulewidth}{0.5pt}
\renewcommand{\footrulewidth}{0.5pt}
\renewcommand{\headrule}{\hbox to\headwidth{\color{headerblue}\leaders\hrule height \headrulewidth\hfill}}
\renewcommand{\footrule}{\hbox to\headwidth{\color{headerblue}\leaders\hrule height \footrulewidth\hfill}}

% Section formatting
\titleformat{\section}{\large\bfseries\color{headerblue}}{\thesection}{1em}{}
\titleformat{\subsection}{\normalsize\bfseries\color{headerblue}}{\thesubsection}{1em}{}

% Table formatting
\newcolumntype{L}[1]{>{\raggedright\arraybackslash}p{#1}}
\newcolumntype{C}[1]{>{\centering\arraybackslash}p{#1}}
\newcolumntype{R}[1]{>{\raggedleft\arraybackslash}p{#1}}

% Reduce line spacing in lists
\setlist{itemsep=2pt,parsep=0pt,topsep=4pt}

\begin{document}

    \title{\textcolor{headerblue}{\textbf{Professional Baguette Formula}}\\
    \large\textcolor{headerblue}{Cold Bulk Fermentation Method with Temperature Controls}}
    \author{Nicolai Guba}
    \date{}
    \maketitle
    \tableofcontents
    \newpage


    \section{Baker's Percentages}

    \subsection{Poolish (Day 1)}
    \begin{table}[h]
        \begin{tabular}{@{}lcc@{}}
            \toprule
            \textbf{Ingredient}    & \textbf{Weight (g)} & \textbf{Baker's \%} \\
            \midrule
            T55 Flour              & 100                 & 20.0\%              \\
            Water                  & 120                 & 24.0\%              \\
            Instant Yeast          & 0.5                 & 0.10\%              \\
            \midrule
            \textbf{Poolish Total} & \textbf{220.2}      & \textbf{44.04\%}    \\
            \bottomrule
        \end{tabular}
    \end{table}

    \subsection{Final Dough}
    \begin{table}[h]
        \begin{tabular}{@{}lcc@{}}
            \toprule
            \textbf{Ingredient}         & \textbf{Weight (g)} & \textbf{Baker's \%} \\
            \midrule
            T55 Flour                   & 500                 & 100.0\%             \\
            Water (total)               & 350                 & 70.0\%              \\
            Salt                        & 10                  & 2.0\%               \\
            Instant Yeast               & 4.5                 & 0.90\%              \\
            Poolish                     & 220.5               & 44.10\%             \\
            \midrule
            \textbf{Total Dough Weight} & \textbf{865.5}      & \textbf{173.10\%}   \\
            \bottomrule
        \end{tabular}
    \end{table}

    \subsection{Component Breakdown}
    \begin{itemize}[leftmargin=*]
        \item \textbf{Flour (total):} 500g (100\%)
        \item \textbf{Water (total):} 350g (70\%)
        \item \textbf{Poolish contribution:} Flour 100g (20\%), Water 120g (24\%)
        \item \textbf{Final dough addition:} Flour 400g, Water 230g
    \end{itemize}


    \section{Temperature Control Matrix}

    \begin{table}[h]
        \small
        \begin{tabular}{@{}cccccc@{}}
            \toprule
            \textbf{Ambient Temp} & \textbf{Poolish Time} & \textbf{Water Temp (°C)} & \textbf{Pre-Bulk} & \textbf{Cold Retard} & \textbf{Proof} \\
            \midrule
            15--17°C              & 18--20h               & 32--35                   & 3.0--4.0h         & 18--36h              & 90--120min     \\
            18--20°C              & 16--18h               & 28--30                   & 2.5--3.0h         & 18--24h              & 75--90min      \\
            21--23°C              & 14--16h               & 24--26                   & 2.0--2.5h         & 12--24h              & 60--75min      \\
            24--26°C              & 12--14h               & 20--22                   & 1.5--2.0h         & 12--18h              & 45--60min      \\
            \bottomrule
        \end{tabular}
    \end{table}

    \section{Production Schedule}

    \subsection{Day 1: Poolish Development}
    \begin{itemize}[leftmargin=*]
        \item \textbf{Time:} T-0 (based on ambient temperature)
        \item \textbf{Storage:} Bowl covered with damp towel (allows gas exchange, prevents skinning)
        \item \textbf{Location:} Counter away from drafts, consistent room temperature
        \item \textbf{Target DDT:} 21--23°C
        \item \textbf{Maturity indicators:} Surface bubbles, slight dome, beginning recession
        \item \textbf{Aroma profile:} Wine-like, mild acetic notes
        \item \textbf{Avoid:} Sealed containers (CO₂ buildup), dry towels (surface skinning), uncovered (drying)
    \end{itemize}

    \subsection{Day 2: Dough Development \& Cold Retard}
    \begin{itemize}[leftmargin=*]
        \item \textbf{Mixing method:} Modified improved mix
        \item \textbf{Target DDT:} 24--25°C (calculated from ambient temp table)
        \item \textbf{Autolyse:} 30--45 minutes (flour + water + poolish only)
        \item \textbf{Mix development:} 4--6 minutes hand kneading to smooth stage
        \item \textbf{Bulk fermentation:} Pre-fridge ambient temperature (see matrix)
        \item \textbf{Fold schedule:} 3--4 folds at 30--45 minute intervals
        \item \textbf{Cold retard:} 4°C for duration per temperature matrix
    \end{itemize}

    \subsection{Day 3: Make-up \& Bake}
    \begin{itemize}[leftmargin=*]
        \item \textbf{Scaling:} 3 units × 288g each
        \item \textbf{Pre-shape:} Minimal degassing, gentle handling
        \item \textbf{Bench rest:} None (cold bulk method)
        \item \textbf{Final shape:} Baguette, 40--45cm length
        \item \textbf{Final proof:} Ambient temperature per matrix
        \item \textbf{Proof test:} Finger poke --- slow partial rebound
    \end{itemize}


    \section{Baking Parameters}

    \subsection{Oven Setup}
    \begin{itemize}[leftmargin=*]
        \item \textbf{Preheat temperature:} 300°C (maximum) with baking metal on top shelf
        \item \textbf{Baking position:} Move metal plate to bottom/lower-middle shelf before loading
        \item \textbf{Deck temperature:} 280°C initial
        \item \textbf{Steam system:} Lava rock method
        \item \textbf{Preheating time:} 60+ minutes minimum
        \item \textbf{Loading method:} Parchment transfer
    \end{itemize}

    \subsection{Baking Profile}

\small
\begin{tabular}{@{}lcccl@{}}
\toprule
\textbf{Stage}  & \textbf{Duration (min)} & \textbf{Temperature} & \textbf{Steam} & \textbf{Notes} \\
\midrule
Initial         & 4                       & 280°C                & Maximum        & Steam generation phase \\
Development     & 7                       & 260°C                & Maintain       & Oven spring period     \\
Steam release   & 1                       & 260°C                & Remove         & Critical transition    \\
Crust formation & 12                      & 260°C                & None           & Extended color development \\
Final browning  & 3                       & 260°C                & None           & Complete at constant temp \\
\bottomrule
\end{tabular}


    \subsection{Quality Standards}
    \begin{itemize}[leftmargin=*]
        \item \textbf{Scoring:} 15° angle, 3--5 cuts, overlapping by ⅓
        \item \textbf{Oven spring:} 30--40\% volume increase
        \item \textbf{Crust color:} Deep golden brown
        \item \textbf{Internal temperature:} 96--99°C
        \item \textbf{Cooling time:} Minimum 60 minutes
    \end{itemize}


    \section{Scaling Formula}

    \textbf{To scale recipe:}
    \begin{itemize}[leftmargin=*]
        \item Multiply all weights by desired factor
        \item Maintain baker's percentages
        \item Adjust timing based on total dough weight:
        \begin{itemize}
            \item 50\% scale: Reduce timing by 10--15\%
            \item 200\% scale: Increase timing by 15--20\%
        \end{itemize}
    \end{itemize}


    \section{Quality Control Points}

    \subsection{Poolish Assessment}
    \begin{itemize}[leftmargin=*]
        \item \textbf{pH target:} 3.8--4.2 (if testing)
        \item \textbf{Specific gravity:} Slight decrease from initial
        \item \textbf{Visual:} Peaked and beginning to recede
    \end{itemize}

    \subsection{Dough Development}
    \begin{itemize}[leftmargin=*]
        \item \textbf{Gluten window:} Minimal required (time develops structure)
        \item \textbf{Temperature monitoring:} Critical for fermentation control
        \item \textbf{Fold response:} Dough should strengthen with each fold
    \end{itemize}

    \subsection{Final Product}
    \begin{itemize}[leftmargin=*]
        \item \textbf{Crumb structure:} Open, irregular holes
        \item \textbf{Crust quality:} Thin, crispy, well-developed ears
        \item \textbf{Flavor profile:} Complex, mild acidity, wheat sweetness
        \item \textbf{Texture:} Chewy crumb, crackling crust
    \end{itemize}

    \vspace{1em}
    \noindent\textbf{Yield:} 3 baguettes × 288g = 864g finished product (2\% bake loss)


    \section{Technique}

    \subsection{Scoring (Just Before Baking)}
    Equipment: Use a lame (curved razor blade holder) or very sharp knife. The blade should be held at a shallow angle (about 30 degrees) with the surface of the loaf, about 0.6 cm/¼ inch deep. \href{https://www.weekendbakery.com/posts/bread-scoring-with-confidence/}{Bread scoring with confidence – Weekend Bakery}

    \begin{itemize}[leftmargin=*]
        \item Make 3--5 diagonal slashes across each baguette, overlapping slightly (about 1/3 overlap)
        \item Cut swiftly and confidently in one smooth motion --- hesitation causes tearing
        \item Each slash should be about 4-5 inches long
        \item The shallow angle creates the classic "ear" that opens during baking
        \item Score just before loading into the oven --- timing is crucial
    \end{itemize}

    \subsection{Baking Process with Lava Rocks}
    Setup (45-60 minutes before baking)
    \begin{itemize}[leftmargin=*]
        \item Place your thick metal plate on bottom shelf
        \item Put lava rocks in a heavy pan on the oven floor or lowest shelf
        \item Preheat to 250-280°C
        \item Use 150-200 ml of hot water
    \end{itemize}
\end{document}